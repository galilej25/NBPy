% Copyright 2004 by Till Tantau <tantau@users.sourceforge.net>.
%
% In principle, this file can be redistributed and/or modified under
% the terms of the GNU Public License, version 2.
%
% However, this file is supposed to be a template to be modified
% for your own needs. For this reason, if you use this file as a
% template and not specifically distribute it as part of a another
% package/program, I grant the extra permission to freely copy and
% modify this file as you see fit and even to delete this copyright
% notice. 

\documentclass[english,aspectratio=169]{beamer}
% Replace the \documentclass declaration above
% with the following two lines to typeset your 
% lecture notes as a handout:
%\documentclass{article}
%\usepackage{beamerarticle}


% There are many different themes available for Beamer. A comprehensive
% list with examples is given here:
% http://deic.uab.es/~iblanes/beamer_gallery/index_by_theme.html
% You can uncomment the themes below if you would like to use a different
% one:
%\usetheme{AnnArbor}
%\usetheme{Antibes}
%\usetheme{Bergen}
%\usetheme{Berkeley}
%\usetheme{Berlin}
%\usetheme{Boadilla}
%\usetheme{boxes}
%\usetheme{CambridgeUS}
%\usetheme{Copenhagen}
%\usetheme{Darmstadt}
%\usetheme{default}
%\usetheme{Frankfurt}
\usetheme{Goettingen}
%\usetheme{Hannover}
%\usetheme{Ilmenau}
%\usetheme{JuanLesPins}
%\usetheme{Luebeck}
%\usetheme{Madrid}
%\usetheme{Malmoe}
%\usetheme{Marburg}
%\usetheme{Montpellier}
%\usetheme{PaloAlto}
%\usetheme{Pittsburgh}
%\usetheme{Rochester}
%\usetheme{Singapore}
%\usetheme{Szeged}
%\usetheme{Warsaw}

\usepackage[utf8x]{inputenc} % Allows Spanish tildes
\usepackage{graphicx} % Allows insert images
\graphicspath{ {fig/} }
\usepackage{ragged2e} % Allows text alignments
\usepackage{hyperref} % Allows hyperlinks
\usepackage{multicol} % Allows multiple columns
%\usepackage{wasysym} % Allows mars and venus symbols
\usepackage[english]{babel}
\usepackage{listings}
%\usepackage{xcolor}

%---------------------------------------------------------------------------------
%	TITLE PAGE
%---------------------------------------------------------------------------------

\title[NBpy]{NBpy: Network-based (R)Statistics in Python}

% A subtitle is optional and this may be deleted
%\subtitle{}

\author[]{Ljuba, Waleed, \& Zeus}
% - Give the names in the same order as the appear in the paper.
% - Use the \inst{?} command only if the authors have different
%   affiliation.


% - Use the \inst command only if there are several affiliations.
% - Keep it simple, no one is interested in your street address.

%\date{}
% - Either use conference name or its abbreviation.
% - Not really informative to the audience, more for people (including
%   yourself) who are reading the slides online

%\logo{\includegraphics[width=\textwidth]{}}

\subject{NBpy}
% This is only inserted into the PDF information catalog. Can be left
% out.

% If you have a file called "university-logo-filename.xxx", where xxx
% is a graphic format that can be processed by latex or pdflatex,
% resp., then you can add a logo as follows:

% \pgfdeclareimage[height=0.5cm]{university-logo}{university-logo-filename}
% \logo{\pgfuseimage{university-logo}}

% Delete this, if you do not want the table of contents to pop up at
% the beginning of each subsection:
%\AtBeginSection[]
%{
%  \begin{frame}<beamer>{Contenidos}
%    \tableofcontents[currentsection]
%  \end{frame}
%}

% Let's get started
\begin{document}

\begin{frame}%1
  \titlepage

  \centering
  \vspace*{-0.5cm}
  \includegraphics[width=\textwidth]{neurohack.png}
\end{frame}

%%%%%%%%%%%%%%%%%%%%%%%%%%%%%%%%%%%%%%%%%%%%%%%%%%%%%%%%%%%%%%%%%%
% Section
\section{Intro}

\begin{frame}{Context}%2
  \framesubtitle{Longitudinal brain networks}

  \centering
  \includegraphics[width=.8\linewidth]{voles.jpg}\\[0.25cm]

  \raggedleft
  (López-Gutierrez et al., \textit{eLife}, 2021)

\end{frame}

\begin{frame}{Context}%3
  \framesubtitle{Prairie voles}

  \centering
  \includegraphics[width=.8\linewidth]{rmvoles.jpeg}\\[0.25cm]

\end{frame}

% Section
\section{NBR}

\begin{frame}{Network Based Statistics framework}%4
  \centering
  \includegraphics[width=\textwidth]{NBR.png}\\[0.1cm]

  \raggedright
  (Gracia-Tabuenca \& Alcauter, \textit{bioRxiv}, 2021)

\end{frame}

\begin{frame}{NBR}%5

  \large
  \begin{itemize}
    \item First CRAN release 0.1.2 (March 2020)
    \begin{itemize}
        \item \url{https://cran.r-project.org/package=NBR}
        \item $>$6k downloads (\url{https://cranlogs.r-pkg.org/badges/grand-total/NBR})
        \item Computing efficiency is the biggest challenge, according to the mails from various users.
    \end{itemize}
  \end{itemize}

\end{frame}

\begin{frame}{NBPy}%6

  \begin{center}
        \textbf{\Large Python implementation}
  \end{center}

\end{frame}

\begin{frame}{R2Py}%7

  \begin{itemize}
    \item \textbf{statsmodels} is a python framework designed for statistics, and allows users to fit statistical models using R-style formulas.
    \item Simply import statsmodels' R-API:
    \includegraphics[width=.8\linewidth]{statsmodels_api.png}\\[0.15cm]
    \item And start R-hacking in Python, with API customization.
  \end{itemize}


\end{frame}


\begin{frame}{statsmodel implementation}
    \begin{itemize}
        \item Statsmodel execution for 1000 randomly selected endog columns finished on average: 3.22 ± 0.73 [ms]
    \end{itemize}
    \begin{center}
        \includegraphics[width=.8\linewidth]{statsmodel_linreg}
    \end{center}

\end{frame}

\begin{frame}{statsmodel}
    \begin{itemize}
        \item Statsmodel execution for 1000 randomly selected endog columns finished on average: 3.22 ± 0.73 [ms]
    \end{itemize}
    \begin{center}
        \includegraphics[width=.8\linewidth]{statsmodel_linreg}
    \end{center}

\end{frame}


\begin{frame}{statsmodel}
    \begin{itemize}
        \item If we run linear regression across all endog columns (interconnections)
    \end{itemize}
    \begin{center}
        \includegraphics[width=.9\linewidth]{statsmodel_columns}
    \end{center}
\end{frame}

\begin{frame}{statsmodel}
    \begin{itemize}
        \item Can we improve performance by computing linear models across all input endog columns, by utilizing matrix computations?
        \item For that reason we will call Pythonic API to handle input data with statsmodels.OLS.
    \end{itemize}

    \begin{center}
        \includegraphics[width=.6\linewidth]{statsmodel_ols}
    \end{center}
\end{frame}

\begin{frame}{statsmodel}
    \begin{itemize}
        \item Not possible to have an endog input with dimension higher than 1.
    \end{itemize}

    \begin{center}
        \includegraphics[width=.9\linewidth]{statsmodel_ols_fail}
    \end{center}
\end{frame}

\begin{frame}{sklearn/numpy}
    \begin{itemize}
        \item sklearn/numpy to the rescue.
        \item We can perform matrix/broadcasting operations on a lowest level of abstraction.
    \end{itemize}

    \begin{center}
        \includegraphics[width=.8\linewidth]{sklearn_linreg}
    \end{center}
\end{frame}

\begin{frame}{sklearn/numpy}
    \begin{itemize}
        \item If we run linear regression across all endog columns (interconnections)
        \item We achieve improvement of calculating linear regression models for all input endog columns from 1.24 seconds to 56.6ms
    \end{itemize}

    \begin{center}
        \includegraphics[width=.8\linewidth]{sklearn_columns}
    \end{center}
\end{frame}

\begin{frame}{sklearn/numpy}
    \begin{itemize}
        \item Although statsmodel has better support for statistical calculations (ie. faster computation of p and t vals), p and t vals calculation is embedeed into package.
    \end{itemize}

    \begin{center}
        \includegraphics[width=.9\linewidth]{conclusion}
    \end{center}
\end{frame}


\begin{frame}{Conclusion}%6

  \begin{center}
        \textbf{\Large Conclusion}
  \end{center}

\end{frame}

\begin{frame}{Conclusion}%6

  \begin{itemize}
      \item R implementation has reach statistical libraries.
      \item On contrary, Python requires implementation of p and t values.
      \item However, it is faster to load bigger datasets with Python and it allows broadcasting/matrix operations.
      \item \textbf{Improvements}:
      \begin{itemize}
          \item Instead of \textbf{sklearn} we can utilize \textbf{numpy.linalg.lstsq} library to compute coefficients of linear regression.
          \item Improve calculation of \textbf{FWE} strength values.
      \end{itemize}
  \end{itemize}

\end{frame}

\end{document}
